\documentclass[a4paper]{article}

\usepackage[english]{babel}
\usepackage[utf8]{inputenc}
\usepackage{amsmath}
\usepackage{graphicx}
\usepackage{hyperref}

\title{AST4320 - Assignment 3}

\author{Metin San}

\date{\today}

\begin{document}
\maketitle


\section*{Exercise 2}
In the lectures we derived that the total optical depth of the ionized IGM due to electron scattering is given by

\begin{equation}\label{eq:tau}
    \tau _ { \mathrm { e } } ( z ) = c \int _ { 0 } ^ { z } \frac { n _ { e } ( z ) \sigma _ { \mathrm { T } } d z } { ( 1 + z ) H ( z ) },
\end{equation}

\noindent where $\sigma _ { \mathrm { T } } = 6.65 \times 10^{-25}$ is the cross-section for electron scattering, $\overline { n } _ { \mathrm { H } } ( z ) \sim 1.9 \times 10 ^
{ - 7 } ( 1 + z ) ^ { 3 } \mathrm{cm}^{-3}$ is the average number density of hydrogen. We have the following cosmological parameters $\Omega _ { \Lambda } = 0.692 , 
\Omega _ { m } = 0.308 \text { and } \Omega _ { r } = 0$. $H(z)$ is the Hubble parameter and is given as

\begin{equation}\label{eq:hub}
    H ^ { 2 } ( z ) = H _ { 0 } ^ { 2 } \left( \Omega _ { m } ( 1 + z ) ^ { 3 } + \Omega _ { r } ( 1 + z ) ^ { 4 } + \Omega _ { \Lambda } \right),
\end{equation}

\noindent where $H_0 \approx 2.19\times 10^{18} \mathrm{s}^{-1}$ is the Hubble parameter today. If we assume that the IGM is completely ionized and consists only of hydrogen,
then $\overline { n } _ { \mathrm { H } } ( z ) \approx n_e(z)$. We are then interested in calculating and plotting the optical depth $\tau_e(z)$ as a function of $z$ in the redshift range $z\in [0,10]$. We do this by creating a \texttt{python} program which solves equation \eqref{eq:tau} for the above values. The program is attached together with the delivery but can also be found on my GitHub\footnote{\href{https://github.com/MetinSa/AST4320/blob/master/Assignment3/Exercise2.py}{\texttt{github.com/metinsa/AST4320/assignment3/exercise2.py}}}. 

The results of the calculation can be seen in figure \ref{fig:tau}. We see that the optical depth today is $\tau_e(z = 0)\approx 0$. We also find $\tau_e(z = 6) = 0.0332$ and $\tau_e(z = 7) = 0.0699$.

\begin{figure}
    \centering
    \includegraphics[width = 0.8\textwidth]{tau.pdf}
    \caption{The optical depth $\tau_e(z)$ of the IGM for redshift $z\in [0,10]$.}
    \label{fig:tau}
\end{figure}

\section*{Exercise 3}
\subsection*{a)}

In the lectures we derived the following second order differential equation for the density
profile of an "isothermal" halo

\begin{equation}\label{eq:halo}
    -\frac{k_b T}{m_{\text{DM}}r^2} \frac{d}{dr} r^2 \frac{d}{dr} \ln{\rho(r)} = 4\pi G \rho(r).
\end{equation}

\noindent We can show that

\begin{equation}\label{eq:halodef}
    \rho(r) = \frac{A}{r^2}, \qquad A = \frac{k_bT}{2\pi G m_{\text{DM}}},
\end{equation}
is a solution to equation \eqref{eq:halo} by substituting it in to the LHS.
We start by rewriting the logarithm expression to the form

\begin{equation*}
    \frac{d \ln(\rho)}{dr} = \frac{1}{\rho} \frac{d \rho}{dr}.
\end{equation*}

\noindent Next up, we insert for $\rho$ and compute the derivative

\begin{equation*}
    \frac{1}{\rho} \frac{d \rho}{dr} = \frac{r^2}{A} \frac{d}{dr} \left( \frac{A}{r^2} \right) = -\frac{2}{r}. 
\end{equation*}

\noindent Inserting this into equation \eqref{eq:halo} and computing the remaining derivative
leaves us with

\begin{equation}\label{eq:halosub}
    \frac{2k_b T}{m_{\text{DM}}r^2} = 4\pi G \rho(r).
\end{equation}

\noindent If we now substitute the constants on the LHS with A, we find

\begin{equation*}
    \frac{k_b T}{m_\text{DM}} = 2\pi G A.
\end{equation*}

\noindent Finally we insert this into \eqref{eq:halosub}

\begin{equation*}
    4\pi G \frac{A}{r^2} = 4\pi G \rho(r).
\end{equation*}

\noindent By reinserting for $\rho(r)$ from definition \eqref{eq:halodef}, we see that this is
indeed the solution.

\subsubsection*{b)}

For an isothermal gas, we have the following equation of state

\begin{equation}\label{eq:isoEOS}
    p = \frac{k_b T}{m_\text{p}} \rho.
\end{equation}

\noindent For a gas in hydrostatic equilibrium we have

\begin{equation}\label{eq:hydrostatic}
    \frac{dp}{dr} = - \frac{GM(<r)\rho}{r^2}.
\end{equation}

\noindent We will now show that an isothermal gas will end up in a similar state. We assume
that the density profile is given similar to that of the isothermal halo, so that 

\begin{equation*}
    \rho (r) = \frac{A_\text{gas}}{r^2}, \qquad  A_\text{gas}  = \frac{k_b T}{2\pi G m_\text{p}}.
\end{equation*}

\noindent Rewriting the pressure in equation \eqref{eq:isoEOS} in terms of $ A_\text{gas} $ results in

\begin{equation}\label{eq:eossub}
    p = 2\pi G  A_\text{gas}  \rho (r).
\end{equation}

\noindent For a spherical symmetric gas, the mass which is smaller than some radius $r$ is
given as

\begin{equation}\label{eq:mass}
    M (<r) = 4\pi \int_0^r x^2 dx \rho(x)  = \frac{4\pi}{3}r^3\rho(r) .
\end{equation}

\noindent We can now rewrite equation \eqref{eq:eossub} in terms of $M $ from equation 
\eqref{eq:mass}. Doing so leaves us with the following equation of state

\begin{equation}\label{eq:eossub2}
    p = \frac{3}{2} \frac{G M (<r) \rho (r)}{r}.
\end{equation}

\noindent Finally, we differentiate equation \eqref{eq:eossub2} with respect to r which 
leaves us with 

\begin{equation}\label{eq:isohydro}
    \frac{dp}{dr} = \frac{-3GM(<r) \rho(r)}{r^2}.
\end{equation}

\noindent We see that this is a very similar to the non-isothermal case seen in equation \eqref{eq:hydrostatic}, the only difference being the factor 3. In both scenarios, the gas results in a state where

\begin{equation}\label{eq:hydroprop}
    \frac{dp}{dr} \propto \frac{M(<r)\rho(r)}{r^2}.
\end{equation}

\end{document}