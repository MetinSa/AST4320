\documentclass[a4paper]{article}
\usepackage[utf8]{inputenc}
\usepackage[T1]{fontenc,url}
\usepackage[english,]{babel}
\usepackage{blindtext}
\usepackage{natbib}
\usepackage{gensymb}
\usepackage{amsmath}
\usepackage{amssymb}
\usepackage{commath}
\usepackage{physics}
\usepackage{multicol}
\usepackage{listings}
\usepackage{graphicx}
\usepackage{hyperref}
\usepackage{svg}
\usepackage{wrapfig}
\urlstyle{sf}



\newenvironment{Figure}
  {\par\medskip\noindent\minipage{\linewidth}}
  {\endminipage\par\medskip}

\def\doubleunderline#1{\underline{\underline{#1}}}
\makeatletter
\newcommand{\xRightarrow}[2][]{\ext@arrow 0359\Rightarrowfill@{#1}{#2}}
\makeatother

 
\usepackage{color}
 
\definecolor{codegreen}{rgb}{0,0.8,0.0}
\definecolor{codegray}{rgb}{0.5,0.5,0.5}
\definecolor{codepurple}{rgb}{0.28,0,0.82}
\definecolor{backcolour}{rgb}{0.95,0.95,0.92}
 
\lstdefinestyle{mystyle}{
    backgroundcolor=\color{backcolour},   
    commentstyle=\color{codegreen},
    keywordstyle=\color{magenta},
    numberstyle=\tiny\color{codegray},
    stringstyle=\color{codepurple},
    basicstyle=\footnotesize,
    breakatwhitespace=false,         
    breaklines=true,                 
    captionpos=b,                    
    keepspaces=true,                                    
    numbersep=5pt,                  
    showspaces=false,                
    showstringspaces=false,
    showtabs=false,                  
    tabsize=2
}
 
\lstset{style=mystyle}
 
\begin{document}
\vspace*{2cm}
\begin{center} 
 
\huge{Assignment 1}

\vspace{15mm}

\large{AST4320: Cosmology and Extragalatic Astronomy}

\vspace{5mm}

\normalsize{Metin San}

\vspace{5mm}

\normalsize{27. August 2018}

\vspace{25mm}

\end{center}

\newpage
\section*{Exercise 1}
\subsection*{(1)}

The continuity equation for an unperturbed universe driven by hubble expansion is given as 
\begin{equation}\label{eq:1}
\frac{d\bar{\rho}}{dt} + \bar{\rho} \nabla \cdot \mathbf{v}_0 = 0,
\end{equation}
where $\bar{\rho}$ is the average density of the universe, and $\mathbf{v}_0$ comes from the Hubble law today which states that 
\begin{equation}\label{eq:2}
\mathbf{v_0} = H_0 \mathbf{r}.
\end{equation}
Here $H_0$ is the Hubble parameter today, given as $H_0 = \dot{a}(t=t_0) / a(t = t_0)$, where $a$ is the scale factor, and is defined so that $a(t= t_0) = 1$. 

Inserting for \eqref{eq:2} into the continuity equation we find
\[
\frac{d\bar{\rho}}{dt} + \bar{\rho} \nabla \cdot \mathbf{r}  \frac{da}{dt} \frac{1}{a}= 0,
\]
where the del operator only works on $\mathbf{r}$ as its the scale factor is only a function of time. The operation results in $\nabla \cdot \mathbf{r} = 3$. We can then swap the right term over to the right hand side of the equation which leaves us with
\[
\frac{d\bar{\rho}}{dt}  = - 3\bar{\rho}\frac{da}{dt} \frac{1}{a}.
\]
Further, we can separate the equation
\[
\frac{d\bar{\rho}}{\rho} = -3 \frac{da}{a}.
\]
This is then solved in the following way
\[
\int_{\bar{\rho}(t=t_0)}^{\bar{\rho}(t)} \frac{d\bar{\rho}}{\rho} = \int_{a(t=t_0)}^{a(t)} -3 \frac{da}{a},
\]
\[
\Rightarrow \ln \left( \frac{\bar{\rho}(t)}{\bar{\rho}(t=t_0)}\right) = \ln \left( \frac{a(t)}{a(t=t_0)}\right)^{-3}.
\]
This can then be reduced to
\begin{equation}\label{eq:3}
\bar{\rho}(t) = \bar{\rho}(t=t_0)a(t)^{-3},
\end{equation}
where we have used that $a(t=t_0) = 1$.


\subsection*{(2)}

The Poisson equation is given as
\begin{equation}\label{eq:4}
\nabla^2 \phi = 4 \pi G \rho.
\end{equation}
We will now study how small perturbations in quantities of interest evolve in time. We introduce the definition $\psi = \psi_0 +\delta \psi$, where $\psi$ is a variable of interest, $\psi_0$ is the unperturbed quantity and $\delta \psi$ is a little perturbation to that variable. The unperturbed Possion equation is then given as

\begin{equation}\label{eq:5}
\nabla^2 \phi_0 = 4 \pi G \rho_0.
\end{equation}
We can apply the perturbation definition to the gravitational potential $\phi$ and density $\rho$ in Poisson's equation, to get
\[
\nabla^2 (\phi_0 + \delta \phi) = 4\pi G (\rho_0 + \delta \rho)
\]
This can be written out on the form
\[
\nabla^2 \phi_0 + \nabla^2 \delta \phi = 4 \pi G \rho_0 +4 \pi G \delta \rho.
\]
We spot that the two terms $\nabla^2 \phi_0$ and $4 \pi G \rho_0$ equal the unperturbed Possion equation. These terms are then removed from the equation as they equal 0. This leaves us with the perturbed Poisson equation
\begin{equation}\label{eq:6}
\nabla^2 \delta \phi = 4 \pi G \delta \rho,
\end{equation}
Which describes how the perturbations evolve in time.
\\

A similar derivation can be carried out for the Euler equation. The equation of motion, or the Euler equation is given as
\begin{equation}\label{eq:7}
    \frac{d \mathbf{v}}{dt} = - \frac{1}{\rho} \nabla p - \nabla \phi.
\end{equation}
In terms of the perturbed quantities, this becomes
\[
    \frac{d (\mathbf{v_0} + \delta \mathbf{v})}{dt} = - \frac{1}{(\rho_0 + \delta \rho)} \nabla (p_0 + \delta p) - \nabla (\phi_0 + \delta \phi),
\]
where 
\begin{equation}\label{eq:8}
    \frac{d\mathbf{v_0}}{dt} = - \frac{1}{\rho_0} \nabla p_0  - \nabla \phi_0
\end{equation}
is the unperturbed equation.
Since the perturbed quantities are very small, we will make the approximation that $1/(\rho_0+ \delta \rho) \approx 1/\rho_0$. In addition to implementing this approximation, we also split up the RHS terms
\[
= - \frac{1}{\rho_0} \nabla p_0 - \frac{1}{\rho_0} \nabla \delta p - \nabla \phi_0 - \nabla \delta \phi.
\]
We spot that the two terms containing the perturbed quantities are equal the RHS of the unperturbed Euler equation \eqref{eq:8}
\section*{Exercise 2}
In the lectures, we sketched how one could arrive at the second order differential equation 

\begin{equation}
\frac{d^2 \delta}{dt^2} + 2 \frac{\dot{a}(t)}{a(t)} \frac{d \delta}{dt} = \delta(4 \pi G \rho_0 - k^2 c_s^2),
\end{equation}
which described the perturbation $\delta(t)$. Here $G$ is the gravitational constant, $k$ is the wavenumber of the perturbation given as $k = 2\pi/ \lambda$, and $c_s$ is the speed of sound in the medium.
\subsection*{(1)}

The Friedman equations can be used to derive the following expression for the Hubble rate or the time evolution of the scale factor 

\begin{equation}
\left(\frac{\dot{a}}{a}\right)^2 = H^2 = H_0^2 \left[ \frac{\Omega_m}{a^3} + \frac{\Omega_r}{a^4} + \Omega_\Lambda \right],
\end{equation}
where $\Omega_i$ is the fractional density parameter and
$i = m,r, \Lambda$ represents the matter, radiation and dark energy contributions to the density. We will then specifically study the three scenarios where we have  $(\Omega_m, \Omega_\Lambda) = (1.0, 0.0)$, $(\Omega_m, \Omega_\Lambda) = (0.3, 0.7)$, and $(\Omega_m, \Omega_\Lambda) = (0.8, 0.2)$. We let $\Omega_r = 0$ for all scenarios as we want to look at the times close to the CMB.

Inserting these numbers into \eqref{eq:7}, we can find expressions for the $\dot{a}/a$ term for all three cases. Doing so gives us the following three expressions

\begin{equation}
\frac{\dot{a}}{a} = H_0 a^{-3/2}
\end{equation}

\begin{equation}
\frac{\dot{a}}{a} = H_0 \left[0.3a^{-3} + 0.7\right]^{1/2}
\end{equation}

\begin{equation}
\frac{\dot{a}}{a} = H_0  \left[0.8a^{-3} + 0.2\right]^{1/2}
\end{equation}
The first expression \eqref{eq:8} corresponds to the Einstein-de Sitter Universe.

\section*{Exercise 4}

We will now consider the non-linear time-evolution of a spherical overdensity in the Einstein-de-Sitter Universe. We assume that this overdensity is confined into a sphere of radius $R(t) \equiv b(t)$, where $b(t)$ is the local scale factor. The acceleration of the radius of this sphere is then given by
\begin{equation}
    \ddot{R} = - \frac{GM}{R^2}.
\end{equation}
We will now show that the following parametrization 

\begin{align}
    R = A(1 - \cos \theta)\\
    t = B(\theta - \sin \theta)\\
    A^3 = GMB^2
\end{align}
satisfies the equation ($\ddot{R}$).

We start by inserting equation ($t$) solved for $B$ into ($A^3$), which gives us
\[
A^3 = \frac{GMt^2}{(\theta - \sin \theta)^2}.
\]
We can further insert this new expression for $A$ into equation ($R$) giving us the following equation for the radius of the sphere
\begin{equation}
    R(t) = \frac{(GM)^{1/3} (1- \cos \theta)}{(\theta - \sin \theta)^{2/3}} t^{2/3}.
\end{equation}
We will then use the first order Taylor approximations $\cos{\theta} \approx 1 - \theta^2/2! $ and $\sin{\theta} \approx \theta - \theta^3/3!$ . Inserting for these into equation $(R)$ we see that
\[
R(t) = \frac{(GM)^{1/3} \left(\frac{\theta^2}{2}\right)}{\left( \frac{\theta^3}{6}\right)^{2/3}} t^{2/3} = (GM)^{1/3}\left(\frac{6^{2/3}}{2}\right) t^{2/3}
\]
the expression becomes independent of the angles. We will now differentiate the angleless expression for the radius twice in order to find the acceleration. Doing so leave us with
\[
    \ddot{R} = - (GM)^{1/3} \left(\frac{6^{2/3}}{2}\right) \left(\frac{2}{9}\right) \frac{1}{t^{4/3}}.
\]
We will now multiply with $R(t)^2$ on both sides of the equation
\[
&   R^2 \ddot{R} = - (GM)^{1/3} \left(\frac{6^{2/3}}{2}\right)               \left(\frac{2}{9}\right) \frac{1}{t^{4/3}} \cdot             (GM)^{2/3}\left(\frac{6^{2/3}}{2}\right)^2 t^{4/3}.
\]
We see that the constant numbers cancel each other, and we are simply left with
\[
R^2 \ddot{R}=  - GM.
\]
By dividing both sides with $R^2$, we end up with the desired results
\begin{equation*}
    \ddot{R} = - \frac{GM}{R^2},
\end{equation*}
which means that the parametrization is satisfied.

\section*{Exercise 5}

We will now compute the infall velocity $v$ from when the material in the overdensity first reaches the viral radius $R_{\text{vir}}$. We will do so by using the parametrization in exercise 4. The infall velocity is given as
\[
    v = \frac{dR}{dt},
\]
where $R$ is the radius from equation (15). By using the chain rule, we can express this as 
\[
    \frac{dR}{dt} = \frac{dR}{d\theta} \frac{d \theta}{dt}.
\]
We start by computing $dR/d\theta$ by differentiating equation (15) with respect to $\theta$
\[
    \frac{dR}{d\theta} = \frac{d}{d\theta} A(1-\cos\theta) = A \sin \theta.
\]
Similarly, we compute $d\theta/dt$ by differentiating expression (16) with respect to $\theta$
\[
    \frac{dt}{d\theta} = \frac{d}{d\theta} B(\theta-\sin\theta) = B(1 - \cos \theta).
\]
and then inverting it. Combining these terms gives us
\[
    v = \frac{A \sin \theta}{B (1- \cos \theta)}.
\]
If we now square each side of the equation, we get
\[
v^2 = \frac{A^2 \sin^2 \theta}{B^2 (1-\cos \theta)^2}.
\]
Further we solve equation (18) for $B^2$ and substitute in
\[
v^2 = \frac{GM\sin^2 \theta}{A(1-\cos \theta)^2}
\]
and then insert for equation (15) solved for $A$, which gives us
\[
    v^2 = \frac{GM \sin^2 \theta}{R (1-\cos \theta)}.
\]
We know that radius equals the virialization radius $R=R_\text{vir}$ when $\theta = 3\pi / 2$. We plug in the virialization angle and find
\[
    v^2 = \frac{GM}{R_\text{vir}}.
\]
Finally, by taking the square root on both sides, we find the following equation for the infall velocity

\begin{equation}
    v = \sqrt{\frac{GM}{R_\text{vir}}}.
\end{equation}
\section*{Exercise 6}

We can derive the gravitational binding energy of a uniform sphere of radius $R$ and mass $M$ by assuming that the density within the sphere i constant. The density is then given as the total mass divided by the volume
\[
\rho = \frac{M}{\frac{4}{3} \pi R^3}.
\]
We then imagine that the sphere is divided into shells with the mass of the outermost shell being
\[
m_{\text{shell}} = 4\pi R^2 \rho\, dr.
\]
The remaining mass within the outer shell is then
\[
m_{\text{interior}} = \frac{4}{3}\pi R^3 \rho.
\]
The energy required for the outer shell to escape the gravitational pull is then given as
\[
dU = -G\frac{m_{\text{shell}} m_{\text{interior}} }{R}.
\]
We find the total energy by inserting for the masses and integrating over all shells
\[
U = -G \int^R_0 \frac{\left(4\pi R^2 \rho\right)  \left(\frac{4}{3}\pi R^3 \rho \right)}{r} dr
\]
\[
= - \frac{16}{3} G\pi^2 \rho^2 \int_0^R R^4 dr
\]
Solving the integral leaves us with 
\[
U = - \frac{16}{15} G \pi^2 \rho^2 R^5.
\]
Finally by inserting for the density, we get

\[
U = - \frac{16}{15} G \pi^2 R^5 \left(\frac{M}{\frac{4}{3} \pi R^3}\right)^2 = - \frac{3}{5} \frac{GM^2}{R},
\]
which was to be shown.
\end{document}