\documentclass[a4paper,11.5pt,]{article}
\usepackage[utf8]{inputenc}
\usepackage[T1]{fontenc,url}
\usepackage[english,]{babel}
\usepackage{blindtext}
\usepackage{natbib}
\usepackage{gensymb}
\usepackage{amsmath}
\usepackage{amssymb}
\usepackage{commath}
\usepackage{physics}
\usepackage{multicol}
\usepackage{listings}
\usepackage{graphicx}
\usepackage{hyperref}
\usepackage{svg}
\usepackage{wrapfig}
\urlstyle{sf}



\newenvironment{Figure}
  {\par\medskip\noindent\minipage{\linewidth}}
  {\endminipage\par\medskip}

\def\doubleunderline#1{\underline{\underline{#1}}}
\makeatletter
\newcommand{\xRightarrow}[2][]{\ext@arrow 0359\Rightarrowfill@{#1}{#2}}
\makeatother

\usepackage{geometry}
 \geometry{
 a4paper,
 total={170mm,257mm},
 textheight =  592mm,
 left=30mm,
 right=30mm,
 tmargin=35mm,
 bmargin=35mm
 }
 
\usepackage{color}
 
\definecolor{codegreen}{rgb}{0,0.8,0.0}
\definecolor{codegray}{rgb}{0.5,0.5,0.5}
\definecolor{codepurple}{rgb}{0.28,0,0.82}
\definecolor{backcolour}{rgb}{0.95,0.95,0.92}
 
\lstdefinestyle{mystyle}{
    backgroundcolor=\color{backcolour},   
    commentstyle=\color{codegreen},
    keywordstyle=\color{magenta},
    numberstyle=\tiny\color{codegray},
    stringstyle=\color{codepurple},
    basicstyle=\footnotesize,
    breakatwhitespace=false,         
    breaklines=true,                 
    captionpos=b,                    
    keepspaces=true,                                    
    numbersep=5pt,                  
    showspaces=false,                
    showstringspaces=false,
    showtabs=false,                  
    tabsize=2
}
 
\lstset{style=mystyle}
 
\begin{document}
\vspace*{2cm}
\begin{center} 
 
\huge{Assignment 1}

\vspace{15mm}

\large{AST4320: Cosmology and Extragalatic Astronomy}

\vspace{5mm}

\normalsize{Metin San}

\vspace{5mm}

\normalsize{27. August 2018}

\vspace{25mm}

\end{center}

\newpage
\section*{Exercise 1}

The continuity equation for an unperturbed universe driven by hubble expansion is given as 
\begin{equation}\label{eq:1}
\frac{d\bar{\rho}}{dt} + \bar{\rho} \nabla \cdot \mathbf{v}_0 = 0,
\end{equation}
where $\bar{\rho}$ is the average density of the universe, and $\mathbf{v}_0$ comes from the Hubble law today which states that 
\begin{equation}\label{eq:2}
\mathbf{v_0} = H_0 \mathbf{r}.
\end{equation}
Here $H_0$ is the Hubble parameter today, given as $H_0 = \dot{a}(t=t_0) / a(t = t_0)$, where $a$ is the scale factor, and is defined so that $a(t= t_0) = 1$. 

Inserting for \eqref{eq:2} into the continuity equation we find
\[
\frac{d\bar{\rho}}{dt} + \bar{\rho} \nabla \cdot \mathbf{r}  \frac{da}{dt} \frac{1}{a}= 0,
\]
where the del operator only works on $\mathbf{r}$ as its the scale factor is only a function of time. The operation results in $\nabla \cdot \mathbf{r} = 3$. We can then swap the right term over to the right hand side of the equation which leaves us with
\[
\frac{d\bar{\rho}}{dt}  = - 3\bar{\rho}\frac{da}{dt} \frac{1}{a}.
\]
Further, we can separate the equation
\[
\frac{d\bar{\rho}}{\rho} = -3 \frac{da}{a}.
\]
This is then solved in the following way
\[
\int_{\bar{\rho}(t=t_0)}^{\bar{\rho}(t)} \frac{d\bar{\rho}}{\rho} = \int_{a(t=t_0)}^{a(t)} -3 \frac{da}{a},
\]
\[
\Rightarrow \ln \left( \frac{\bar{\rho}(t)}{\bar{\rho}(t=t_0)}\right) = \ln \left( \frac{a(t)}{a(t=t_0)}\right)^{-3}.
\]
This can then be reduced to
\begin{equation}\label{eq:3}
\bar{\rho}(t) = \bar{\rho}(t=t_0)a(t)^{-3},
\end{equation}
where we have used that $a(t=t_0) = 1$.


\section*{Exercise 2}

The unperturbed Poisson equation is given as

\begin{equation}\label{eq:4}
\nabla^2 \phi_0 = 4 \pi G \rho_0.
\end{equation}
By rewriting in terms of the perturbation terms, we get the following equation
\[
\nabla^2 (\phi + \delta \phi) = 4\pi G (\rho + \delta \rho)
\]
We can then write this out to the form
\[
\nabla^2 \phi + \nabla^2 \delta \phi = 4 \pi G \rho +4 \pi G \delta \rho.
\]
Since the two terms $\nabla^2 \phi$ and $4 \pi G \rho$ simply makes up the Possion equation which is equal to 0, we are left with the perturbed Possion equation
\begin{equation}\label{eq:4}
\nabla^2 \delta \phi = 4 \pi G \delta \rho.
\end{equation}

\end{document}