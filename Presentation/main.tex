\documentclass{article}
\usepackage[utf8]{inputenc}

\title{Presentation - AGN / Quasars}
\author{Metin San}
\date{November 2018}

\begin{document}

\maketitle

\section*{Content}
This presentation will consist of a quick introduction to what an AGN is. We will follow that up by the discovery of the first AGN and Quasar. We will then go deeper into what a Quasar actually is and how it works. Then we will dive deeper into certain types of AGN and how one usually studies these before we finish up with the topic of AGN Unifciation which the modern interpretation of AGN.


\section*{Introduction}
Sometimes when we observe galaxies, we find that their centeral region is much more luminous than that of the average galaxy. This suggests that there are high activity going on there, so we have named this phenomena AGN ( Active Galactic Nucleus). There are many different types of AGNs, for instance the Quasar which we will touch on shortly.

\section*{Discovery}
When the first radio telescopes pointed their scopes towards the sky, they quickly noticed something wierd. They found huge radioblobs. The blobbyness was a result of the low precision of the telescopes, but their source was quickly detected, and the object was named 3C 273 (Typical nice astro name). So astronomers did what we always do, they took the spectra of the object. To their surprise, we found that the spectra looked nowhere near that of known stars. (Side note: If we were to resolve this spectra, we would see the famous Lyman alpha forest which we have discussed in the lectures). By further study of the spectra they found that the object had a redshift of z = 0.158 which would but object 3C 273 a whole 2Gly away! They decided to name this object a Quasi Stellar Radio Source, also known as a Quasar.

\section*{What is a Quasar?}
Theories quickly began to rise to what this object could be. Was it perhaps a swarm of neutron stars? Or maybe it was an alien civilization which was capable of harnessing and controlling the energy of their entire galaxy? But by the 1980s astronomers converged towards what i feel is the coolest explanation.

Take a black hole the size of millions to billions of solar masses. Where would we find such a supermassive black hole? Turns out nearly every decent sized galaxy has one at their center. When then imagine that gas from the host galaxy (Galaxy containing the SMBH) and drive it towards the BH. As the gas decends towards the black hole, it is accelerated to 1000s of km/s and is heated immensly by friction. Imagine yourself rubbing hands. Now do the same thing at a 1000 km/s. Thats gonna get hot. The gas forms what is called an accretion disk, and the heat glow of this disk can be seen from nearly anywhere in the Universe.

Much of the gas which falls towards the black hole never crosses the event horizon. This is rather converted directly to energy by the friction in terms of photons, and these photons can result in powerful Quasar winds which drive gas back into the galaxy and acts as feedback which we touched on in the lectures. In some Quasars some of the gas is collimated and channeled relativistic jets which erupts from their poles. The physics behind these jets are still discussed, but it is likely a result of rapidly spinning black holes with powerfull magnetic fields. If these jets point right towards us, the AGN is called a Blazar.

\section*{Types of AGN and how to study them}
As you might have understood by now, AGNs come in different forms. There are actually so many different types of AGNs that have been categorized over the years that we call this situation the AGN zoo. We dont have time to look at all the different types, but we will cover some, with focus on how Quasars appear in the different bandwidhts.

\subsection*{Radio Selection}
Sometimes a jetted quasar is seen to blast through the entire galaxy creating beautiful radio plumes. These galaxies are called Radio Galaxies, and are usually dominated by radiation in the Radio bandwidth. This radio emission is thought to result from synchrotron radiation. Syncrhotron radiation occurs when relativistic particles are accelerated in a magnetic fields. Radio galaxies are further separated into many sub groups, some which are RF I, RF II, Radio Loud and Radio Quiet. Since synchroton radiation is rare and usually only appears from supernova remains, we can also be sure that the Signal is directly from the AGN.

Radio quiet and loud are defined by the taking the ratio between the radio and optical flux. If this is greater than 10, then the galaxy is Radio loud, and if not, radio quiet. We see that the spectral energy of these two groups are mostly similar with Radio Loud galaxies having in general higher energies.

Two other popular subgroups of Radio AGN is the Fanaroff and Riley types.
FR I types are categorized by having core dominated emission, where the jets fade quickly, while FR II are in general more luminous with longer extending jets and are edge bright. We see examples of these in the plot. A more modern categorization is simply the Jetted and Not Jetted division of radio galaxies. This is implemented as many of the subgroups have huge overlaps and are often misleading. We will come back to this overlap later.

\subsection*{IR Selection}
Studying AGNs in the Infra red spectrum is also interesting. In general, SMBHs have a dusty torus as seen in the plot. It is thought that the IR signal one recieves originates from reproccesed photons which traveled trough the torus. We can therefore learn alot about the blackholes surroundings by studying the IR signal. 

A problem when studying the IR bandwidth is that IR signal often originates from standard stellar source such as stars. It can therefore be a bit hard to differentiate between the signal originating from the AGN to its surroundings.

\subsection*{Optical-UV Selection}
The optical bandwidth is perhaps the most important bandwidth we have for studying AGN. It allows us to take the spectrum of the AGNs which can be used to determine interesting features such as its composition and redshift. The AGNs most commonly observed AGNs in the Optical-UV bandwidth is standard Quasars and Seyfert galaxies.

Seyfert galaxies are AGNs with Quasar like nuclei, the difference being that we are able to resolve their host galaxy. This is usually because we find them at low redshifts. They look like normal galaxies at first glance, but when studied in different bandwidths it becomes clear that they are AGNs. They are general a bit weaker than the average Quasar.

The Optical bandwidth selection is further important as it can provide us with interesting quantities such as the Eddington ratio which can be used to determine the mass of the Black hole. The black hole mass function is very complicated, so we will just state the results from this method, and that is that most AGNs have a typical mass of $10^6 - 10^{10}$ solar masses.

One should be carefull about contamination in Optical as Galaxy emmisions are usually high in Optical, also depending on the redshift of the Quasar, the color of stellar Optical signal can mimic that of the AGN signal.

\subsection*{X-ray and Gamma-ray Selection}
Studying AGNs at the highest energies can also provide useful information. X-rays and Gamma-rays usually penetrate mostly everything in their path which means that the signal contains little to no contamination. The gamma-rays originate from the jets and are usually only seen in blazars, while the X-rays originate from both the accretion disk and the yet and is caused by inverse Compton scattering. 

Since X-rays are mostly unabsorbed, the amount of absorption gives a good estimate of the density surroundings. A important quantity to describe this is the Column density which is basically how many hydrogen atoms are in the specific line of sight between observed and object given as $N_H$. We also find that most of the X-ray signal from AGNs are universal, meaning that they appear the same independant of AGN.

\section*{Unification of AGN}
The more and more we study AGN, the more overlap between categories we find. This fact in addition to the universality of the non absorbed emissions suggest that most AGN probably are more alike than we think. More and more studies suggests that the most important parameters are the Quasars orientation with respect to us, wether or not it has jets, the power of its accretion disk, the optical thickness and shape of its torus, at which stage the AGN is in its time evolution, and a couple more parameters.

There is still much to learn before we are able to find a unifying theory. We will likely not find a complete unification, but its clear that the historical categorization and the AGN zoo is simply a result of variations to the above parameters.

\section*{Conclusion}
AGNs are some of the most powerful non explosive phenomena in astrophsyics. They appear differently depending on specific parameteres and the selction of bandwidth which one uses to observe them.

Studying AGNs is important as they play a role in many astrophysical contexts. The light from a Quasar helps us understand the Intergalactic medium through the Lyman alpha forest, and understanding the Quasar wind can give us insight into star formation and galaxy evolution.




\end{document}

